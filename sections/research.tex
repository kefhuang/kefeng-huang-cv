%====================
% Plant Bot 
%====================
\subsection{
    Robotic System for Identifying and Watering House Plants 
    \hfill 10/2021 --- 04/2022
}
    \subtext{Supervised by Prof. Simon Julier at University College London}
    \begin{zitemize}
        \item   Proved the possibility of an autonomous robotic system to water houseplants
                by creating one in a simulated environment in ROS Noetic, 
                with a group of 6 people working in different aspects including
                SLAM, Control, Path Planing, Classification \& Detection and Exploration.
                Available on Github repository
                \url{https://github.com/Aashvin/COMP0031-PlantBot}.
        \item   \textbf{Classification \& Detection}: 
                Conducted a literature review on detection and 
                segmentation methods as well as the plant identification networks.
                Integrated Darknet\_ROS package into our project.
                Modified the package to support different YOLO versions including YOLOv4.
                Available on Github repository
                \url{https://github.com/t1mkhuan9/yolov4-ros-noetic}.
        \item   \textbf{Score:}
                81 in individual literature review, 72 in group report and 80 in individual report
    \end{zitemize}


%====================
% Image Seg
%====================
\subsection{
    Learning 3D Point Cloud Segmentation by Aggregating 2D Image Semantics
    \hfill 08/2021 --- 12/2021
}
    \subtext{Self-motivated Research Project}
    \begin{zitemize}
        \item   Projected each 3D point to corresponding image pixels in different frames
                and used image semantics information to generate 3D point segmentation.
                Explored two possible approaches:
                (1) directly using the result produced by image segmentation by choosing
                the mode among them
                (2) removed the final layer of the image segmentation network and use the
                features to get 3D point information.
        \item   Conducted experiments based on the KITTI-Odometry dataset using the 
                ground truth value from SemanticKITTI dataset.
                Used and adapted Nvidia Segmentation (DeepV3WPlus Network) as the image
                segmentation method.
                Trained a simple neural network based on the image semantics features to
                predict the 3D point semantic label.
        \item   Available on Github repository: \url{https://github.com/t1mkhuan9/image-based-pointcloud-segmentation}
    \end{zitemize}
