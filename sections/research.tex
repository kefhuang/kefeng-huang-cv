
%====================
% Plant Bot 
%====================
\subsection{{Robotic System for Identifying and Watering House Plants 
    \hfill 10/2021 --- 04/2022}}
\subtext{Supervised by Prof. Simon Julier at University College London}
\begin{zitemize}
    \item \textbf{Brief}: 
        This project is to create an autonomous robotic system to water houseplants
        in a simulated environment in ROS Noetic. 
        We were working in a group of 6 people with different aspects including
        SLAM, Control, Path Planing, Classification \& Detection and Exploration.
        The project repository is public on Github repository
        \url{https://github.com/Aashvin/COMP0031-PlantBot}.
    \item \textbf{Personal Contribution}: 
        My work was about Classification \& Detection. 
        A 3-month literature review is first conducted on detection and 
        segmentation methods as well as the plant identification networks.
        Darknet\_ROS package was then integrated to our project. 
        To support YOLOv4, the package was then modified and the new package 
        created can be found on Github repository
        \url{https://github.com/t1mkhuan9/yolov4-ros-noetic}.

\end{zitemize}

%====================
% Image Seg
%====================
\subsection{{Image Based Point Cloud Segmentation 
    \hfill 08/2021 --- Present}}
\subtext{Guided by PhD student Henry at University of Oxford}
\begin{zitemize}
    \item \textbf{Technology}: Experiments are run over KITTI-Odometry dataset 
        as it contains both images and point cloud in each frame.
        SemanticKITTI is also used to provide the labels.
        Python is the main programming language as well as Pytorch to train
        the network.
        Nvidia Segmentation (DeepV3WPlus) is adapted to extract features from 
        images.
    \item \textbf{Description}: Each 3D point is first projected to images within
        the same frame and the frames before. 
        It is first implemented to choose the mode of the labels of the same point
        on different images. 
        Then it is improved by making predictions using the features of the images
        instead.
        A network is trained on a new dataset that combines the actual point
        label and the corresponding images features. 
\end{zitemize}
